\documentclass{article}

\usepackage{amsmath, amsthm, amssymb, amsfonts}
\usepackage{thmtools}
\usepackage{graphicx}
\usepackage{setspace}
\usepackage{geometry}
\usepackage{float}
\usepackage{hyperref}
\usepackage{bbold}
\usepackage[utf8]{inputenc}
\usepackage[english]{babel}
\usepackage{framed}
\usepackage[dvipsnames]{xcolor}
\usepackage{tcolorbox}
\usepackage{xcolor}
\usepackage{notes}  % Include the custom style file
% ------------------------------------------------------------------------------

\begin{document}

% ------------------------------------------------------------------------------
% Cover Page and ToC
% ------------------------------------------------------------------------------

\title{ \normalsize \textsc{}
		\\ [2.0cm]
		\HRule{1.5pt} \\
		\LARGE \textbf{\textit{Asymptotic Statistics} \\ \Large{by VAN DER VAART}
		\HRule{2.0pt} \\ \vspace*{10\baselineskip}}
		}
\date{5/24/2024 - ?}
\author{Scribed by:
\textbf{Sida Li}}

\maketitle
\newpage

\tableofcontents
\newpage

\section{Prelim: Probability and Measure}
\begin{note}
    This part follows Chapter 1 in Keener's \textit{Theoretical Statistics} textbook. It is meant to introduce some basic concepts that will be assumed throughout rest of the note. I also bring out necessary notations from here. \textbf{Precise definitions are usually ommitted, but can be found in the textbook.}
\end{note}

\subsection{Measure and Probability Space}

We start by refreshing the rigorous definition of \textbf{measure space}, then proceed with examples.

Given a set $\mathcal{X}$, a $\sigma$-\textbf{algebra} $\mathcal{A}$ is a collection of subsets of $\mathcal{X}$ that (1) contains $\mathcal{X}$ and the empty set $\emptyset$, (2) is closed under complements, and (3) is closed under countable unions. 

A \textbf{measure} $\mu:\mathcal{A} \rightarrow [0, \infty]$ is a function that assigns a non-negative real number to each element in $\mathcal{A}$, such that $\mu(\emptyset) = 0$ and $\mu$ is countably additive, i.e. for any disjoint sequence of sets $\{A_i\}_{i=1}^{\infty}$, 
\begin{equation}
    \mu\left(\bigcup_{i=1}^{\infty} A_i\right) = \sum_{i=1}^{\infty} \mu(A_i)
\end{equation}
The triple $(\mathcal{X}, \mathcal{A}, \mu)$ is called a \textbf{measure space}. A measure $\nu$ such that $\nu(\mathcal{X}) = 1$ is called a \textbf{probability measure}, and then we can define a \textbf{probability space} $(\mathcal{X}, \mathcal{A}, \nu)$.

\subsection{Lebesgue Measure and Why We Care}

Many statistics textbook, including Keener's, assures the readers that ``measure theory is not needed'' -- the motivation behind this simplification is that most of the time, we are working with the \textbf{Lebesgue measure} on $\mathbb{R}^d$. Lebesgue measure is compatible with the usual notion of length, area, and volume (in 1, 2, and 3-dimensions), and is defined on the \textbf{Borel $\sigma$-algebra} $\mathcal{B}(\mathbb{R}^d)$, which is the smallest $\sigma$-algebra containing all open sets in $\mathbb{R}^d$. 

Another benefits of Lebesgue measure is its connection with our usual notion of integration. If a function $f$ is \textbf{Lebesgue integrable} (whose definition is detailed \href{https://math.stackexchange.com/questions/1716526/what-does-it-mean-to-say-that-a-function-is-integrable-with-respect-to-a-measure}{here}), then its integral is:
\begin{equation}
    \int_{\mathbb{R}^d} f(x)\, d\mu(x) = \int_{\mathbb{R}^d} f(x)\, dx
\end{equation}
where the RHS is the familiar Riemann integral. Throughout this notes, we consider all measures $\nu$ on $\mathbb{R}^d$\footnote{we also assume $\nu$ to be $\sigma$-finite, whose definition is skipped} to be \textbf{absolutely continuous} with respect to the Lebesgue measure $\mu$, i.e. $\nu(A) = 0$ implies $\mu(A) = 0$. Then by \textbf{Radon–Nikodym's theorem}, we can write
\begin{equation}
    \int_{\mathbb{R}^d} f(x)\, d\nu(x) = \int_{\mathbb{R}^d} f(x) \frac{d\nu}{d\mu}(x)\, d\mu(x) = \int_{\mathbb{R}^d} f(x)\nu(x) \,dx
\end{equation}
for $\nu$-integrable functions $f$. Hopefully this clarifies why we can safely ignore measure theory in most of the (basic) statistics literature.


\newpage
\section{Convergence of Random Variables}


\begin{note}
	This part is largely following Chapter 2 of the book, with focus on the first three subsections.
\end{note}
% ------------------------------------------------------------------------------


% ------------------------------------------------------------------------------
% Reference and Cited Works
% ------------------------------------------------------------------------------

% \bibliographystyle{apalike}
% \bibliography{References.bib}

% ------------------------------------------------------------------------------

\end{document}
